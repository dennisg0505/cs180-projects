\title{CS 180 Homework 4}
\date{\today}

\documentclass[12pt]{article}
\usepackage[margin=1in]{geometry}
\usepackage{enumerate}
\usepackage{scrextend}
\usepackage{filecontents,tikz}
\usepackage{pgfplots}
\usepackage{listings}
\usepackage{graphicx}
\usepackage{svg}
\lstset{tabsize=2}
\pgfplotsset{compat=1.7}

\begin{document}
\maketitle

\section{Dynamic Coin Grabbing}\label{coin grabbing}
\begin{enumerate}[(a)]
\item Show a sequence of $n \geq 6$ coins for which it is not optimal for the first player to start by picking up the available coin of larger value. That is, give an example for which the natural greedy strategy is suboptimal.
\begin{addmargin}[2em]{2em}

\end{addmargin}
\item Give an $O(n^2)$ algorithm to compute an optimal strategy for the first player. Given the initial sequence,
your algorithm should precompute some information in $O(n^2)$ time, and then the first player should be
able to make each move optimally in $O(1)$ time by looking up the precomputed information.
\begin{addmargin}[2em]{2em}
\begin{lstlisting}

\end{lstlisting}
\end{addmargin}
\item Find a sequence of $n \geq 6$ coins for which the dynamic programming strategy guarantees the first player a higher total value than the simple strategy
\begin{addmargin}[2em]{2em}

\end{addmargin}
\end{enumerate}

\section{Box Stacking}\label{box stacking}
\begin{enumerate}[(a)]
\item Give an algorithm for box stacking that uses as few split and join steps as possible, when given two stacks of boxes � which are represented as ascendingly sorted sequences of numbers (box sizes). Duplicate numbers are permitted in the input sequences, and the two stacks can contain identical numbers. Hint: the goal is to find the optimal split of one of two stacks.
\begin{addmargin}[2em]{2em}
\begin{lstlisting}

\end{lstlisting}
\end{addmargin}
\item Determine the time complexity of your algorithm.
\begin{addmargin}[4em]{2em}

\end{addmargin}
\end{enumerate}

\section{Optimal Submatrix}\label{optimal submatrix}
\begin{enumerate}[(a)]
\item Find a maximal positive rectangular submatrix - i.e., a submatrix containing only positive values that has the most elements.
\begin{addmargin}[2em]{2em}
\begin{lstlisting}
Function maximumPositiveMatrix(Matrix)
	Let MostPositive = 0
	For each Left from 0 to columns(Matrix)
		Let Temp be an integer array with rows(Matrix) zeroes
		For each Right from Let to columns(Matrix)
			For each i from 0 to rows(Matrix)
				Temp[i] += Matrix[i][Right]
			Let NumPositive = maximumPositiveArray(Temp, MaxStart, MaxTail)
			If NumPositive > MostPositive
				MostPositive = NumPositive
				MaxRowStart = Left;
				MaxRowTail = Right;
				MaxColStart = maxStartForRij;
				MaxColTail = maxTailForRij;

	Return Matrix with corners MaxRowStart, MaxRowTail, 
		MaxColStart, MaxColTail

Function maximumPositiveArray(Array, byref(Start), byref(End))
	Let MostPositive = 0
	Let NumPositive = 0
	Let Start = 0
	Let End = -1
	Let Index = 0

	For each i from 0 to sizeof(Array)
		NumPositive = max(0, NumPositive + Array[x])
		MostPostive = max(MaxPositive, NumPositive)
		If MostPostive = NumPositive
			End = i
		Else
			Start = End + 1

	Return MostPositive
\end{lstlisting}
Note: rows() is the number of rows of a matrix and columns() is the number of columns of a matrix.
\end{addmargin}
\item Find a maximum sum rectangular submatrix - i.e., a submatrix whose elements have maximal sum. (Hint: This is a generalization of the `maxsum' problem discussed at th
e start of this course.).
\begin{addmargin}[2em]{2em}
\begin{lstlisting}
Function maximumSumMatrix(Matrix)
	Let MaxSum = 0
	For each Left from 0 to columns(Matrix)
		Let Temp be an integer array with rows(Matrix) zeroes
		For each Right from Let to columns(Matrix)
			For each i from 0 to rows(Matrix)
				Temp[i] += Matrix[i][Right]
			Let Sum = maximumSumArray(Temp, MaxStart, MaxTail)
			If Sum > MaxSum
				MaxSum = Sum
				MaxRowStart = Left;
				MaxRowTail = Right;
				MaxColStart = maxStartForRij;
				MaxColTail = maxTailForRij;

	Return Matrix with corners MaxRowStart, MaxRowTail, 
		MaxColStart, MaxColTail
\end{lstlisting}
Note: maximumSumArray() is Kadane's Algorithm (that returns the maxsum and updates seconds and third parameters to indicate the location of the maximum subarray), rows() is the number of rows of a matrix, and columns() is the number of columns of a matrix.
\end{addmargin}
\end{enumerate}

\section{Going Beyond the Master Theorem}\label{going beyond the master theorem}
\begin{enumerate}[(a)]
\item 
\begin{addmargin}[2em]{2em}
The recurrence can be solved via a recursion tree as follows. First we can rewire $n$ as $n = 2^{\log n}$. With each recursive call downward, the square root will be taken. With some arbitrary $z$ iterations of the recursion, the $z$-th root of $n$ is taken as follows: $n^{1/2^{k}} = 2^{{\log n}/2^{k}}$. If we terminate when this expression is less than two, we logically reach $2^{{\log n}/2^{k}} = 2$, which can be rewritten as ${\log n}/2^{k} = 1$ by taken the $\log$ of both sides. This can further be simplified to  $\log n = 2^{k}$, which after taking the $\log$ of both sides again reduces to $\log\log n = k$. With each recursion tree step doing $n$ work, the total is $O(n \log\log n)$ which is exactly the big-O expected.
\end{addmargin}
\item 
\begin{addmargin}[2em]{2em}
$T(n) = T(2^p)$ \\ \\
\begin{tabular}{l || l | r}
$p$ & $T(2^p)$ & Value \\
\hline
0 & $T(1) = 1$ & 1  \\
1 & $T(2) = 8$ & 8  \\
2 & $T(4) = 3*8 + 4*1 + 3*4 = 40$ & 40  \\
3 & $T(8) = 3*40 + 4*8 + 3*8 = 176$ & 176  \\
4 & $T(16) = 3*176 + 4*40 + 3*16 = 736$ & 736  \\
5 & $T(32) = 3*736 + 4*176 + 3*32 = 3008$ & 3008  \\
\end{tabular} \\ \\
Therefore, $T(2^p) = 2^p(3*2^p-2)$. Furthermore, the integer $k$ for which $T(n) = \Theta(n^k)$ is $k = 0$.
\end{addmargin}
\item 
\begin{addmargin}[2em]{2em}
First, it can be shown that $T(n)$ is equal to the sum $\sum_{k=1}^{n}\frac{1}{2k-1}$ simply through iteration: \\ \\
$T(n) = T(n-1) + \frac{1}{2n-1}$ \\
$T(n) = T(n-2) + \frac{1}{2n-3} + \frac{1}{2n-1}$ \\
$T(n) = T(n-3) + \frac{1}{2n-5} + \frac{1}{2n-3} + \frac{1}{2n-1}$ \\
$T(n) = \frac{1}{2n-1} + \frac{1}{2n-3} + \frac{1}{2n-5} + ...$ \\
$T(n) = \sum_{k=1}^{n}\frac{1}{2k-1}$ \\ \\
Next, the sum can be expressed as the difference of two sums as follows: $\sum_{k=1}^{n}\frac{1}{2k-1} = \sum_{p=1}^{2n-1}\frac{1}{p} - (...)$. The first part is simply the harmonic series, which via calculus yields a result of $\log n$ for a series from 1 to $n$. However, given a series to $2n-1$, the solution becomes $\log \sqrt{n}$. The second part takes $O(1)$ constant time, resulting in a solution of $\log \sqrt{n} + O(1)$, where $\log$ is the natural logarithm.
\end{addmargin}
\item 
\begin{addmargin}[2em]{2em}
We begin with the following derivative, $\sum_{k=1}^{n}kc^{k-1} = \frac{d}{dc}\sum_{k=1}^{n}c^{k}$. From this, we can simply expand the known exponential sum as $\frac{d}{dc}\sum_{k=1}^{n}c^{k} = \frac{d}{dc}(\frac{c(c^n-1)}{c-1})$, which further yields (via derivation) $\frac{nc^{n+1} - (n + 1)c^n + 1}{{c-1}^2}$. Resulting in the answer we expected of $\sum_{k=1}^{n}kc^{k-1} = \frac{nc^{n+1} - (n + 1)c^n + 1}{{c-1}^2}$.
\end{addmargin}
\end{enumerate}
\end{document}