\title{CS 180 Homework 5}
\date{\today}

\documentclass[12pt]{article}
\usepackage[margin=1in]{geometry}
\usepackage{enumerate}
\usepackage{scrextend}
\usepackage{filecontents,tikz}
\usepackage{pgfplots}
\usepackage{listings}
\usepackage{graphicx}
\usepackage{svg}
\lstset{tabsize=2}
\pgfplotsset{compat=1.7}

\begin{document}
\maketitle

\section{Mobile Computing}\label{mobile computing}
Design a polynomial-time algorithm for the following problem: Given the positions of a set of clients and a set of base stations, decide whether every client can be connected simultaneously to a base station, subject to the range and load constraints.
\begin{addmargin}[2em]{2em}

\end{addmargin}

\section{Simple Maximum Flow}\label{max flow}
\begin{enumerate}[(a)]
\item Find a maximum flow; how many planeloads of people can we move from MDL to LAX?
\begin{addmargin}[2em]{2em}

\end{addmargin}
\item Find a minimum cut for this network, and give its capacity.
\begin{addmargin}[4em]{2em}

\end{addmargin}
\item Is the minimum cut unique? (Is there any other cut with the same capacity?)
\begin{addmargin}[4em]{2em}

\end{addmargin}
\end{enumerate}

\section{Min Flow / Max Cut}\label{min flow}
\begin{enumerate}[(a)]
\item Give a polynomial time algorithm that finds the minimum possible flow on $G$.
\begin{addmargin}[2em]{2em}

\end{addmargin}
\item Consider any minimum flow $f$. Is it true that it corresponds to some ‘maximum cut’ ? That is, for networks like this, is there a Min Flow / Max Cut theorem? If so, prove it; if not, give a counterexample.
\begin{addmargin}[2em]{2em}

\end{addmargin}
\end{enumerate}

\section{Directed Hamiltonian Path}\label{hamiltonian path}
Show that the Directed Hamiltonian Path problem is NP-complete. Show that this problem is in NP, then show that it is NP-hard by reduction from some other problem
\begin{addmargin}[2em]{2em}

\end{addmargin}
\end{document}