\title{CS 180 Homework 3}
\date{\today}

\documentclass[12pt]{article}
\usepackage[margin=1in]{geometry}
\usepackage{enumerate}
\usepackage{scrextend}
\usepackage{filecontents,tikz}
\usepackage{pgfplots}
\usepackage{listings}
\usepackage{graphicx}
\usepackage{svg}
\pgfplotsset{compat=1.7}

\begin{document}
\maketitle

\section{Rectangles}\label{rectangles}
\begin{enumerate}[(a)]
\item Design an $O(n \log n)$ algorithm that finds the outline of the rectangles.
\begin{addmargin}[2em]{2em}
\begin{lstlisting}
Let I be the array of rectangle arrays
Sort array I by the first element of each rectangle array in ascending order with mergesort
Let O be an empty array of integers
For each array R in I
	If R[1] > nextR[1]
		Push R[0] to O


\end{lstlisting}
\end{addmargin}
\item Design an O(n) algorithm that, given an outline, finds a rectangle of maximal area that fits within the outline. Implement your algorithm with a single left-to-right scan through the outline data.
\begin{addmargin}[2em]{2em}
\begin{lstlisting}
Let O be the array representing the outline
Let S, i be 0
While O[i+2] is not null
	S += (O[i+2] - O[i]) / O[i+1]
	i += 2
Return S
\end{lstlisting}
\end{addmargin}
\end{enumerate}

\section{Interview Questions}\label{interview questions}
\begin{enumerate}[(a)]
\item You are given a 3-pint container and a 5-pint container, and as much water as you want. Specify a sequence of filling and emptying steps that leave the containers holding exactly 7 pints of water.
\begin{addmargin}[2em]{2em}
The first step is to fill up the the 5-pint container and empty as much as possible into the 2-pint container. This leaves you with 3 pints and 2 pints in the two containers. The next step is to empty the full 3-pint container and pour the 2 pints from the 5-pint container into the 3-pint container. Finally, you fill the empty 5-pint container, leaving us with 2 pints in the 3-pint container and 5 pints in the 5-pint container for a total of 7 pints of water
\end{addmargin}
\item There are many problems like the one above; it is from the 1916 Stanford-Binet IQ test. A common interview question uses 3, 5, 4. Design an algorithm that, given three small integers like these as input, finds a sequence with the minimum number of steps.
\begin{addmargin}[2em]{2em}
\begin{lstlisting}

\end{lstlisting}
\end{addmargin}
\item You are given an array $A$ of $n$ integers, and another integer $z$, and you want to determine
whether the array contains two elements $a$ and $b$ such that $a + b = z$.
\begin{addmargin}[2em]{2em}
\begin{enumerate}[i.]
\item Give an algorithm that uses a min-heap and a max-heap to determine this in time $O(n \log n)$.
\begin{addmargin}[4em]{2em}

\end{addmargin}
\item Give an algorithm that runs in time $O(n)$, assuming that $A$ is given to you in sorted order.
\begin{addmargin}[4em]{2em}
\begin{lstlisting}
Let A be the array of integers
Let i be 0
Let j be n-1
While i < j
	Sum = A[i] + A[j]
	If Sum > z
		j -= 1
	Else if Sum < z
		i += 1
	Else
		Return true
Return false
\end{lstlisting}
\end{addmargin}
\end{enumerate}
\end{addmargin}
\item You are given an array $A$ of $n$ integers (possibly negative) and you want to determine whether the array contains three elements $a$, $b$, and $c$ such that $a+b+c = 0$. Give an algorithm that solves this problem in $O(n^2)$ time.
\begin{addmargin}[2em]{2em}
\begin{lstlisting}
Let A be the array of integers
Use merge-sort to sort in ascending order
For i from 0 to n-1
	j = i+1
	k = n-1
	While j < k
		Sum = A[i] + A[j] + A[k]
		If Sum > 0
			k -= 1
		Else if Sum < 0
			j += 1
		Else
			Return true
Return false
\end{lstlisting}
\end{addmargin}
\item You are given an array of size $n$ containing every number in ${0,1,2,...,n}$ except for one. Give an algorithm to find the missing number in time $O(n)$, using only 1 memory cell that has $\lceil 2 \log_{2}{n} \rceil$ bits. (For example, when $n = 50000$, the cell has 32 bits, and can represent numbers from $0$ to $2^{32} -���1$.)
\begin{addmargin}[2em]{2em}
\begin{lstlisting}
Let I be the input array
Let CELL be the memory cell
CELL = 0
For each i from 1 to n-1
	CELL = CELL XOR I[i]
For each j from 0 to n-1
	CELL = CELL XOR j
Return CELL
\end{lstlisting}
\end{addmargin}
\end{enumerate}

\section{Optimal Submatrix}\label{optimal submatrix}
\begin{enumerate}[(a)]
\item Find a maximal positive rectangular submatrix - i.e., a submatrix containing only positive values that has the most elements.
\begin{addmargin}[2em]{2em}

\end{addmargin}
\item Find a maximum sum rectangular submatrix - i.e., a submatrix whose elements have maximal sum. (Hint: This is a generalization of the `maxsum' problem discussed at the start of this course.).
\begin{addmargin}[2em]{2em}

\end{addmargin}
\end{enumerate}

\section{Going Beyond the Master Theorem}\label{going beyond the master theorem}
\begin{enumerate}[(a)]
\item 
\begin{addmargin}[2em]{2em}

\end{addmargin}
\end{enumerate}
\end{document}